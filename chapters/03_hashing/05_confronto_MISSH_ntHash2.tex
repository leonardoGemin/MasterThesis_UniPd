\section{MISSH vs ntHash2}
\label{sec:confronto_MISSH_ntHash2}

%\acs{MISSH} e ntHash2 sono entrambi strumenti innovativi progettati per un hashing efficiente delle sequenze nucleotidiche, ognuno dei quali offre approcci e vantaggi unici nelle proprie implementazioni.
\acs{MISSH} and ntHash2 are both innovative tools designed for efficient hashing of nucleotide sequences, each offering unique approaches and advantages in their implementations.

%\acs{MISSH}, introdotto come miglioramento degli algoritmi FSH e \acs{ISSH}, incorpora nuove strategie per gestire simultaneamente più semi spaziati. L'approccio ISSH Multi elabora simultaneamente diversi semi spaziati, ottimizzando l'efficienza dell'hashing attraverso la costruzione di una matrice di hashing organizzata per colonne. Questo design facilita il riutilizzo degli hash precedentemente calcolati, riducendo significativamente il carico di lavoro computazionale e migliorando la velocità di elaborazione complessiva. Inoltre, ISSH Multi Column perfeziona ulteriormente questo concetto introducendo un meccanismo per recuperare le posizioni da hash associati a semi con spazi diversi, migliorando la flessibilità e l'adattabilità dell'algoritmo. L'analisi delle prestazioni dimostra sostanziali accelerazioni in varie configurazioni di semi e lunghezze di lettura, dimostrando l'efficacia di \acs{MISSH} nell'accelerare le operazioni di calcolo degli hash.
\acs{MISSH}, introduced as an enhancement to the FSH and \acs{ISSH} algorithms, incorporates new strategies for handling multiple spaced seeds simultaneously. The ISSH Multi approach processes several spaced seeds simultaneously, optimising hashing efficiency by constructing a hashing matrix organised by columns. This design facilitates the reuse of previously calculated hashes, significantly reducing the computational workload and improving the overall processing speed. Furthermore, ISSH Multi Column further refines this concept by introducing a mechanism to retrieve positions from hashes associated with differently spaced seeds, improving the flexibility and adaptability of the algorithm. Performance analysis demonstrates substantial speed-ups in various configurations of seeds and read lengths, proving the effectiveness of \acs{MISSH} in accelerating hash computation operations.

%Al contrario, ntHash2 si basa sulla funzione di hashing ricorsivo di ntHash, introducendo miglioramenti su misura per l'hashing a semi distanziati. ntHash2 rivoluziona il processo di hashing incorporando una funzione di rotazione divisa e ridefinendo la formula di hashing per accogliere in modo efficiente i semi distanziati. Utilizzando l'hashing a blocchi e ottimizzando il calcolo dei valori hash, ntHash2 ottiene notevoli miglioramenti delle prestazioni, superando il suo predecessore ntHash e altri algoritmi concorrenti come CityHash e \acs{ISSH}. La versatilità e la scalabilità dell'algoritmo lo rendono adatto a diverse applicazioni come l'assemblaggio di genomi e il conteggio di k-mer, offrendo vantaggi significativi in termini di velocità e precisione.
In contrast, ntHash2 builds upon the recursive hashing function of ntHash, introducing enhancements tailored for spaced seed hashing. ntHash2 revolutionizes the hashing process by incorporating a split rotation function and redefining the hashing formula to accommodate spaced seeds efficiently. By utilizing block-based hashing and optimizing hash value calculation, ntHash2 achieves remarkable performance improvements, outperforming its predecessor ntHash and other competing algorithms like CityHash and \acs{ISSH} \cite{mohamadi2016ntHash}. The algorithm's versatility and scalability make it suitable for diverse applications such as genome assembly and k-mer counting, offering significant advantages in speed and accuracy.


