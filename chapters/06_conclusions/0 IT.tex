\chapter{Conclusions}
\label{chp:conclusions}

Il progetto DuoHash ha portato allo sviluppo di uno strumento altamente efficiente per l'hashing di spaced $k$-mer per sequenze nucleotidiche, migliorando significativamente le prestazioni rispetto agli strumenti esistenti come ntHash2. I test di performance hanno dimostrato che DuoHash offre un notevole incremento della velocità, con uno speed-up massimo osservato di circa $11\times$ rispetto a ntHash2. Speed-up elevati sono particolarmente evidenti nei dataset caratterizzati da reads di lunghezza costante, dove DuoHash ha mostrato una superiorità marcata. Questi risultati sono stati ottenuti grazie a innovazioni significative nell'architettura dei dati, come l'utilizzo di strutture di hash ottimizzate e tabelle di look-up pre-computate, che hanno permesso di migliorare drasticamente le performance di calcolo. La flessibilità del sistema, con la possibilità di modificare facilmente la funzione di hash e di gestire spaced seed sia simmetrici che asimmetrici, ha ampliato le potenzialità applicative di DuoHash, rendendolo versatile per diversi tipi di analisi genomiche. Ad esempio, l'implementazione della funzione \verb|getSpacedKmer| e la compatibilità con JellyFish hanno reso DuoHash uno strumento potente per la pre-elaborazione delle sequenze nucleotidiche per il conteggio di spaced $k$-mer. Questi contributi sono significativi per il campo della bioinformatica, in quanto offrono una soluzione che non solo migliora le prestazioni, ma amplia anche le possibilità di analisi attraverso un approccio più flessibile e integrato. 

Insieme ai risultati promettenti, ci sono alcuni aspetti di DuoHash che potrebbero essere ulteriormente migliorati. In particolare, l'implementazione di tecniche di parallelizzazione più avanzate potrebbe migliorare ulteriormente le prestazioni. Sebbene DuoHash supporti già l'esecuzione parallela attraverso il metodo \verb|DuoHash_parallel|, esistono ulteriori margini per ottimizzare questo aspetto, sfruttando al meglio le risorse hardware disponibili e riducendo i tempi di calcolo. Una possibile strada potrebbe essere la revisione della struttura di salvataggio dati per permettere l'implementazione di tecniche come le istruzioni \ac{SIMD}. Attualmente, la struttura di salvataggio dei dati di DuoHash non consente l'efficace utilizzo di istruzioni. Ripensare l'architettura dei dati in modo da allinearla con i requisiti delle istruzioni \acs{SIMD} potrebbe quindi rappresentare un significativo miglioramento. Questo potrebbe comportare la riorganizzazione dei dati in blocchi contigui in memoria, ottimizzati per l'accesso parallelo e la computazione simultanea, minimizzando così il tempo di latenza e massimizzando il throughput. Questi miglioramenti renderebbero DuoHash ancora più competitivo e versatile, permettendo di gestire dataset di dimensioni sempre maggiori e di adattarsi a diverse esigenze di ricerca. In sintesi, mentre DuoHash rappresenta già un significativo passo avanti nel campo dell'hashing delle sequenze genomiche, l'adozione di tecniche di parallelizzazione avanzata e l'ottimizzazione delle strategie di esecuzione parallela sono direzioni promettenti per il futuro sviluppo di DuoHash, che potrebbero portare a ulteriori miglioramenti delle prestazioni e a una maggiore efficienza nell'analisi delle sequenze genomiche.
