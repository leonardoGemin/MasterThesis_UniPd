\chapter{Conclusions}
\label{chp:conclusions}

The DuoHash project led to the development of a highly efficient tool for hashing spaced $k$-mer for nucleotide sequences, significantly improving performance over existing tools such as ntHash2. Performance tests have shown that DuoHash offers a significant speed-up, with an observed maximum speed-up of about $11\times$ compared to ntHash2. High speed-ups are particularly evident in datasets characterised by reads of constant length, where DuoHash showed marked superiority. These results were achieved thanks to significant innovations in the data architecture, such as the use of optimised hash structures and pre-computed look-up tables, which drastically improved the calculation performance. The flexibility of the system, with the possibility of easily modifying the hash function and handling both symmetric and asymmetric spaced seeds, has expanded the application potential of DuoHash, making it versatile for different types of genomic analysis. For instance, the implementation of the \verb|getSpacedKmer| function and compatibility with JellyFish have made DuoHash a powerful tool for pre-processing nucleotide sequences for spaced $k$-mer counting. These contributions are significant for the field of bioinformatics, as they offer a solution that not only improves performance, but also expands analysis possibilities through a more flexible and integrated approach. 

Along with the promising results, there are some aspects of DuoHash that could be further improved. In particular, the implementation of more advanced parallelisation techniques could further improve performance. Although DuoHash already supports parallel execution through the \verb|DuoHash_parallel| method, there is further scope for optimising this aspect, making better use of the available hardware resources and reducing computation times. One possible avenue could be the revision of the data-saving structure to allow the implementation of techniques such as \ac{SIMD} instructions. Currently, the data storage structure of DuoHash does not allow the effective use of instructions. Rethinking the data architecture to align it with the requirements of instructions \acs{SIMD} could therefore be a significant improvement. This could involve reorganising data into contiguous blocks in memory, optimised for parallel access and concurrent computation, thus minimising latency time and maximising throughput. These improvements would make DuoHash even more competitive and versatile, allowing it to handle ever larger datasets and to adapt to different research needs. In summary, while DuoHash already represents a significant step forward in the field of hashing genomic sequences, the adoption of advanced parallelisation techniques and the optimisation of parallel execution strategies are promising directions for the future development of DuoHash, which could lead to further performance improvements and greater efficiency in the analysis of genomic sequences.
