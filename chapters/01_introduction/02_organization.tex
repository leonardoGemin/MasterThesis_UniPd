\section{Organization of the work}
\label{sec:thesis-organization}

The structure of the thesis is organised into the following chapters:

\begin{description}
	\item[Chapter \ref{chp:spacedkmer}: \nameref{chp:spacedkmer}.] In this chapter, DNA concepts, sequencing methods and assembly techniques are introduced. We introduce $k$-mer and spaced $k$-mer, their applications in bioinformatics, including advantages and disadvantages.
	\item[Chapter \ref{chp:hashing}: \nameref{chp:hashing}.] In this chapter, various hashing methods for spaced $k$-mer are described, including the tools \ac{FSH}, \ac{ISSH} and \ac{MISSH}, ntHash and ntHash2.
	\item[Chapter \ref{chp:develop}: \nameref{chp:develop}.] This chapter introduces the DuoHash tool, the new version of MISSH, and discusses the new features introduced.
	\item[Chapter \ref{chp:results}: \nameref{chp:results}.] In this chapter, the temporal performance of the new tool is analysed across various datasets and experimental configurations, comparing the performance of DuoHash with ntHash2. The performance of DuoHash and its integration with JellyFish compared to third-party tools such as MaskJelly are also analysed.
	\item[Chapter \ref{chp:conclusions}: \nameref{chp:conclusions}.] This chapter summarises the results obtained, discussing the practical implications and proposing future directions for research.
	\item[Appendices.] Additional tables complete with times and speed-ups, and graphs, comparing DuoHash with ntHash2 and integrating DuoHash with JellyFish are provided.
\end{description}

This organisation aims to guide the reader through a thorough understanding of the problem, the proposed solutions and their experimental evaluations, culminating in a summary of conclusions and potential future research directions.