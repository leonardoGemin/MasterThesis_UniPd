\section{Purpose of the thesis}
\label{sec:thesis-purpose}

L'obiettivo principale di questa tesi è sviluppare un software avanzato che calcoli in modo efficiente il forward hashing e il reverse complement hashing per spaced k-mer nelle sequenze nucleotidiche. Questo software è pensato per migliorare la gestione e l'analisi delle grandi quantità di dati genetici generate dalle moderne tecniche di sequenziamento del DNA. Un altro obiettivo cruciale è ottimizzare la velocità e l'efficienza computazionale del processo di hashing. Considerando la crescente dimensione dei dataset genomici, è fondamentale disporre di strumenti che possano eseguire analisi rapide senza compromettere l'accuratezza. Il software sviluppato dovrà essere in grado di gestire questi volumi di dati in modo efficiente, riducendo al minimo i tempi di elaborazione e le risorse computazionali necessarie. Inoltre, il software dovrà essere testato su vari dataset per valutare le sue performance rispetto agli strumenti esistenti. Questo include un'analisi dettagliata dei tempi di esecuzione e un confronto con altre tecniche di hashing attualmente in uso.