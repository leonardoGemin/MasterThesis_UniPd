\section{Organization of the work}
\label{sec:thesis-organization}

La struttura della tesi è organizzata nei seguenti capitoli:

\begin{description}
	\item[Chapter \ref{chp:spacedkmer}: \nameref{chp:spacedkmer}.] In questo capitolo si introducono i concetti di DNA, i metodi di sequenziamento e le tecniche di assemblaggio. Si introducono $k$-mer e spaced $k$-mer, le loro applicazioni in bioinformatica, comprensive di vantaggi e svantaggi.
	\item[Chapter \ref{chp:hashing}: \nameref{chp:hashing}.] In questo capitolo si descrivono vari metodi di hashing per gli spaced $k$-mer, inclusi i tools \ac{FSH}, \ac{ISSH} e \ac{MISSH}, ntHash e ntHash2.
	\item[Chapter \ref{chp:develop}: \nameref{chp:develop}.] Questo capitolo presenta il tool DuoHash, la nuova versione di MISSH, e discute le nuove funzionalità introdotte.
	\item[Chapter \ref{chp:results}: \nameref{chp:results}.] In questo capitolo si analizzano le performance temporali del nuovo strumento attraverso vari dataset e configurazioni sperimentali, confrontando le prestazioni di DuoHash con ntHash2. Vengono inoltre analizzate le performance di DuoHash e la sua integrazione con JellyFish rispetto a tool terzi come MaskJelly.
	\item[Chapter \ref{chp:conclusions}: \nameref{chp:conclusions}.] Questo capitolo riassume i risultati ottenuti, discutendo le implicazioni pratiche e proponendo direzioni future per la ricerca.
	\item[Appendices.] Vengono fornite tabelle aggiuntive complete di tempi e speed-up, e grafici, relativi alla comparazione di DuoHash con ntHash2 e all'integrazione di DuoHash con JellyFish.
\end{description}

Questa organizzazione mira a guidare il lettore attraverso una comprensione approfondita del problema, delle soluzioni proposte e delle loro valutazioni sperimentali, culminando con una sintesi delle conclusioni e delle potenziali direzioni future della ricerca.