\section[Spaced $k$-mer]{Spaced k-mer}
\label{sec:spacedkmer}

Gli spaced seed ---~e, di conseguenza, gli spaced $k$-mer~--- rappresentano strumenti fondamentali nella bioinformatica, particolarmente nel contesto dell'allineamento delle sequenze e della ricerca di similitudini tra DNA, RNA e proteine \cite{brinda2015spaced,ounit2016higher,girotto2017theoretical}. A differenza dei $k$-mer tradizionali, gli spaced seed consentono di introdurre dei gap (posizioni ignorate) all'interno della sequenza di nucleotidi o amminoacidi, permettendo così un confronto più flessibile delle sequenze. In modo formale, uno spaced seed $Q$ è una stringa sull'alfabeto $\{0, 1\}$, dove gli \texttt{1} corrispondono alle matching positions: \[ Q = \left\{ x \; | \; x \in \{0, 1\}^*, \; \#_1(x) = w \right\}, \] dove $k$ è la \emph{lunghezza}, o \emph{span}, e $w$ è il \emph{peso} dello spaced seed. Dato che qualsiasi posizione \texttt{0} posta prima del primo \texttt{1} e qualsiasi posizione \texttt{0} posta dopo l'ultimo \texttt{1} non cambia lo spaced seed, consideriamo solamente gli spaced seed che iniziano e terminano con il carattere \texttt{1}, definiti dalla seguente espressione regolare: \[ Q = \left\{ x \; | \; x = 1 + 1 \cdot (0 + 1)^* \cdot 1, \; \#_1(x) = w \right\}. \]
Si possono rappresentare gli spaced seed anche attraverso la loro \emph{forma} $Q$, che è l'insieme delle posizioni degli \texttt{1} nello spaced seed \cite{keich2004spaced}. In questo caso il peso di $Q$ è definito come $k = |Q|$, mentre la lunghezza è pari a $s(Q) = \max(Q) + 1$. 

Gli spaced $k$-mer, chiamati anche $Q$-gram, sono fragmenti di una sequenza nucleotidica $x$ che rispettano lo schema dettato da uno spaced seed $Q$. Data una stringa $x$, lo spaced $k$-mer $x[i + Q]$ è una stringa così definita: \[ x[i + Q] = \{ x_{i + k} \; | \; k \in Q \}, \] dove $i \in \{0, 1, \dots, |x| - s(Q) \}$.

Consideriamo lo spaced seed \texttt{1011011}, definito come $Q = \{ 0, 2, 3, 5, 6 \}$, e la sequenza vista nella sezione precedente, \[ \texttt{CTTGTCGTTGGAGATCGGAAGAGC}, \] i suoi $|x| - s(Q) + 1 = 18$ $Q$-gram sono: \[ \begin{matrix}
	x[0 + Q] = \texttt{CTGCG} & x[1 + Q] = \texttt{TGTGT} & x[2 + Q] = \texttt{TTCTT} \\
	x[3 + Q] = \texttt{GCGTG} & x[4 + Q] = \texttt{TGTGG} & x[5 + Q] = \texttt{CTTGA} \\
	x[6 + Q] = \texttt{GTGAG} & x[7 + Q] = \texttt{TGGGA} & x[8 + Q] = \texttt{TGAAT} \\
	x[9 + Q] = \texttt{GAGTC} & x[10 + Q] = \texttt{GGACG} & x[11 + Q] = \texttt{AATGG} \\
	x[12 + Q] = \texttt{GTCGA} & x[13 + Q] = \texttt{ACGAA} & x[14 + Q] = \texttt{TGGAG} \\
	x[15 + Q] = \texttt{CGAGA} & x[16 + Q] = \texttt{GAAAG} & x[17 + Q] = \texttt{GAGGC} \\
\end{matrix} \]

%Uno dei principali vantaggi degli spaced seed è la loro capacità di migliorare la sensibilità dell'allineamento delle sequenze senza compromettere significativamente la specificità. Questo è particolarmente utile in scenari in cui le sequenze possono presentare variazioni locali, come piccole mutazioni o indel. Tuttavia, l'implementazione di algoritmi che lavorano con gli spaced seed è, generalmente, computazionalmente più intensiva rispetto ai $k$-mer contigui a causa della complessità aggiuntiva nella gestione dei gap. 
%
%Gli spaced seed sono ampiamente utilizzati in diversi campi della bioinformatica. Un'applicazione comune è nel miglioramento degli algoritmi di allineamento delle sequenze, come BLAST e suoi derivati, dove l'uso di spaced seed ha dimostrato di incrementare sia la sensibilità che la velocità di ricerca \cite{bin2002patternhunter,li2004patternhunter}. Nella ricerca di omologie, algoritmi come PatternHunter utilizzano spaced seed per migliorare l'efficienza del confronto tra sequenze genomiche di diverse specie, riducendo il numero di falsi positivi rispetto ai metodi basati su $k$-mer contigui \cite{li2004patternhunter}. Nell'assemblaggio \emph{de novo} dei genomi, gli spaced $k$-mer possono essere utilizzati per costruire grafi di De Bruijn più robusti, che facilitano il riconoscimento di contigui corretti nonostante la presenza di errori di sequenziamento \cite{nagarajan2009parametric}. Inoltre, nel campo del metagenomica, l'uso di spaced seed consente un'analisi più accurata delle comunità microbiche, migliorando la distinzione tra specie simili e l'identificazione di nuove varianti genomiche \cite{burkhardt2006enhanced}.

Uno dei principali vantaggi degli spaced seed è la loro capacità di migliorare la sensibilità dell'allineamento delle sequenze senza compromettere significativamente la specificità. Questo è particolarmente utile in scenari in cui le sequenze possono presentare variazioni locali, come piccole mutazioni o indel. Gli spaced seed, infatti, permettono di catturare regioni di omologia anche quando sono presenti discontinuità o variazioni, che potrebbero sfuggire agli approcci basati su $k$-mer contigui \cite{mak2006improvements}. Tuttavia, l'implementazione di algoritmi che lavorano con gli spaced seed è, generalmente, computazionalmente più intensiva rispetto ai $k$-mer contigui a causa della complessità aggiuntiva nella gestione dei gap: la progettazione e l'ottimizzazione degli spaced seed richiede una maggiore elaborazione per determinare le posizioni ottimali dei gap, e il calcolo delle corrispondenze diventa più complesso \cite{chumakov2012bioinformatics}. Nonostante queste sfide, l'uso di spaced seed è giustificato dai miglioramenti significativi in termini di sensibilità e accuratezza.

Gli spaced seed sono ampiamente utilizzati in diversi campi della bioinformatica. Un'applicazione comune è nel miglioramento degli algoritmi di allineamento delle sequenze, come BLAST e suoi derivati, dove l'uso di spaced seed ha dimostrato di incrementare sia la sensibilità che la velocità di ricerca \cite{bin2002patternhunter,li2004patternhunter}. L'introduzione degli spaced seed in BLAST ha permesso di migliorare la rilevazione di regioni omologhe in sequenze biologiche, riducendo il numero di false positive e migliorando l'efficienza complessiva del processo di allineamento \cite{camacho2009blast+}.

Nella ricerca di omologie, algoritmi come PatternHunter utilizzano spaced seed per migliorare l'efficienza del confronto tra sequenze genomiche di diverse specie, riducendo il numero di falsi positivi rispetto ai metodi basati su $k$-mer contigui \cite{li2004patternhunter}. Questo approccio è particolarmente utile nella comparazione di genomi complessi, dove le variazioni locali possono mascherare le somiglianze a livello globale. Inoltre, l'uso di spaced seed consente di identificare omologie remote che potrebbero non essere rilevate con metodi più tradizionali \cite{sun2005enhanced}.

Nell'assemblaggio \emph{de novo} dei genomi, gli spaced $k$-mer possono essere utilizzati per costruire grafi di De Bruijn più robusti, che facilitano il riconoscimento di contigui corretti nonostante la presenza di errori di sequenziamento \cite{nagarajan2009parametric}. Gli spaced seed permettono di migliorare la continuità e la completezza degli assemblaggi genomici, riducendo gli artefatti causati da errori sistematici e randomici nella lettura delle sequenze \cite{paten2011genome}.

Inoltre, nel campo della metagenomica, l'uso di spaced seed consente un'analisi più accurata delle comunità microbiche, migliorando la distinzione tra specie simili e l'identificazione di nuove varianti genomiche \cite{burkhardt2006enhanced}. Questo è particolarmente importante in ambienti complessi come il suolo o l'intestino umano, dove la diversità microbica è elevata e la risoluzione fine delle specie è cruciale per comprendere le dinamiche ecologiche e funzionali delle comunità microbiche \cite{turnbaugh2007human}.

In conclusione, gli spaced seed rappresentano un avanzamento significativo nella bioinformatica, offrendo un equilibrio tra sensibilità e specificità in una varietà di applicazioni. Sebbene la loro implementazione possa richiedere risorse computazionali maggiori, i benefici ottenuti nella precisione dell'analisi delle sequenze giustificano ampiamente il loro utilizzo.
