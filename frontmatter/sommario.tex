\begin{abstract}[it]
	Questa tesi si concentra sul miglioramento dell'estrazione e della codifica hash di $k$-mer spaziati per applicazioni bioinformatiche. Esplora il concetto di semi spaziati, che migliorano il rilevamento della somiglianza consentendo corrispondenze non consecutive all'interno dei $k$-mer, anche se a spese di una maggiore complessità computazionale.
	
	Lo scopo principale di questa ricerca è sviluppare un software avanzato in grado di eseguire rapidamente l'hashing e l'hashing del complemento inverso per i $k$-mer spaziati nelle sequenze nucleotidiche. Ciò include l'ottimizzazione del processo di hashing per gestire meglio grandi insiemi di dati genomici e minimizzare il tempo di elaborazione e le risorse computazionali. Il lavoro include l'introduzione dello strumento DuoHash, una versione migliorata di \ac{MISSH} e ne confrontiamo le prestazioni con ntHash2. I risultati dimostrano come DuoHash si comporta su diversi set di dati, mostrando la sua efficienza in termini di tempo e l'integrabilità con strumenti come JellyFish. Infine, vengono discusse le implicazioni pratiche e i suggerimenti per le future direzioni di ricerca.
\end{abstract}